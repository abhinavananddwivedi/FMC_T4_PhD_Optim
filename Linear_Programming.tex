\documentclass[11pt,]{article}
\usepackage{lmodern}
\usepackage{amssymb,amsmath}
\usepackage{ifxetex,ifluatex}
\usepackage{fixltx2e} % provides \textsubscript
\ifnum 0\ifxetex 1\fi\ifluatex 1\fi=0 % if pdftex
  \usepackage[T1]{fontenc}
  \usepackage[utf8]{inputenc}
\else % if luatex or xelatex
  \ifxetex
    \usepackage{mathspec}
  \else
    \usepackage{fontspec}
  \fi
  \defaultfontfeatures{Ligatures=TeX,Scale=MatchLowercase}
\fi
% use upquote if available, for straight quotes in verbatim environments
\IfFileExists{upquote.sty}{\usepackage{upquote}}{}
% use microtype if available
\IfFileExists{microtype.sty}{%
\usepackage{microtype}
\UseMicrotypeSet[protrusion]{basicmath} % disable protrusion for tt fonts
}{}
\usepackage[margin = 1.5in]{geometry}
\usepackage{hyperref}
\PassOptionsToPackage{usenames,dvipsnames}{color} % color is loaded by hyperref
\hypersetup{unicode=true,
            pdftitle={Linear Programming},
            pdfauthor={Abhinav Anand},
            colorlinks=true,
            linkcolor=blue,
            citecolor=magenta,
            urlcolor=red,
            breaklinks=true}
\urlstyle{same}  % don't use monospace font for urls
\usepackage{color}
\usepackage{fancyvrb}
\newcommand{\VerbBar}{|}
\newcommand{\VERB}{\Verb[commandchars=\\\{\}]}
\DefineVerbatimEnvironment{Highlighting}{Verbatim}{commandchars=\\\{\}}
% Add ',fontsize=\small' for more characters per line
\usepackage{framed}
\definecolor{shadecolor}{RGB}{248,248,248}
\newenvironment{Shaded}{\begin{snugshade}}{\end{snugshade}}
\newcommand{\KeywordTok}[1]{\textcolor[rgb]{0.13,0.29,0.53}{\textbf{#1}}}
\newcommand{\DataTypeTok}[1]{\textcolor[rgb]{0.13,0.29,0.53}{#1}}
\newcommand{\DecValTok}[1]{\textcolor[rgb]{0.00,0.00,0.81}{#1}}
\newcommand{\BaseNTok}[1]{\textcolor[rgb]{0.00,0.00,0.81}{#1}}
\newcommand{\FloatTok}[1]{\textcolor[rgb]{0.00,0.00,0.81}{#1}}
\newcommand{\ConstantTok}[1]{\textcolor[rgb]{0.00,0.00,0.00}{#1}}
\newcommand{\CharTok}[1]{\textcolor[rgb]{0.31,0.60,0.02}{#1}}
\newcommand{\SpecialCharTok}[1]{\textcolor[rgb]{0.00,0.00,0.00}{#1}}
\newcommand{\StringTok}[1]{\textcolor[rgb]{0.31,0.60,0.02}{#1}}
\newcommand{\VerbatimStringTok}[1]{\textcolor[rgb]{0.31,0.60,0.02}{#1}}
\newcommand{\SpecialStringTok}[1]{\textcolor[rgb]{0.31,0.60,0.02}{#1}}
\newcommand{\ImportTok}[1]{#1}
\newcommand{\CommentTok}[1]{\textcolor[rgb]{0.56,0.35,0.01}{\textit{#1}}}
\newcommand{\DocumentationTok}[1]{\textcolor[rgb]{0.56,0.35,0.01}{\textbf{\textit{#1}}}}
\newcommand{\AnnotationTok}[1]{\textcolor[rgb]{0.56,0.35,0.01}{\textbf{\textit{#1}}}}
\newcommand{\CommentVarTok}[1]{\textcolor[rgb]{0.56,0.35,0.01}{\textbf{\textit{#1}}}}
\newcommand{\OtherTok}[1]{\textcolor[rgb]{0.56,0.35,0.01}{#1}}
\newcommand{\FunctionTok}[1]{\textcolor[rgb]{0.00,0.00,0.00}{#1}}
\newcommand{\VariableTok}[1]{\textcolor[rgb]{0.00,0.00,0.00}{#1}}
\newcommand{\ControlFlowTok}[1]{\textcolor[rgb]{0.13,0.29,0.53}{\textbf{#1}}}
\newcommand{\OperatorTok}[1]{\textcolor[rgb]{0.81,0.36,0.00}{\textbf{#1}}}
\newcommand{\BuiltInTok}[1]{#1}
\newcommand{\ExtensionTok}[1]{#1}
\newcommand{\PreprocessorTok}[1]{\textcolor[rgb]{0.56,0.35,0.01}{\textit{#1}}}
\newcommand{\AttributeTok}[1]{\textcolor[rgb]{0.77,0.63,0.00}{#1}}
\newcommand{\RegionMarkerTok}[1]{#1}
\newcommand{\InformationTok}[1]{\textcolor[rgb]{0.56,0.35,0.01}{\textbf{\textit{#1}}}}
\newcommand{\WarningTok}[1]{\textcolor[rgb]{0.56,0.35,0.01}{\textbf{\textit{#1}}}}
\newcommand{\AlertTok}[1]{\textcolor[rgb]{0.94,0.16,0.16}{#1}}
\newcommand{\ErrorTok}[1]{\textcolor[rgb]{0.64,0.00,0.00}{\textbf{#1}}}
\newcommand{\NormalTok}[1]{#1}
\usepackage{graphicx,grffile}
\makeatletter
\def\maxwidth{\ifdim\Gin@nat@width>\linewidth\linewidth\else\Gin@nat@width\fi}
\def\maxheight{\ifdim\Gin@nat@height>\textheight\textheight\else\Gin@nat@height\fi}
\makeatother
% Scale images if necessary, so that they will not overflow the page
% margins by default, and it is still possible to overwrite the defaults
% using explicit options in \includegraphics[width, height, ...]{}
\setkeys{Gin}{width=\maxwidth,height=\maxheight,keepaspectratio}
\IfFileExists{parskip.sty}{%
\usepackage{parskip}
}{% else
\setlength{\parindent}{0pt}
\setlength{\parskip}{6pt plus 2pt minus 1pt}
}
\setlength{\emergencystretch}{3em}  % prevent overfull lines
\providecommand{\tightlist}{%
  \setlength{\itemsep}{0pt}\setlength{\parskip}{0pt}}
\setcounter{secnumdepth}{0}
% Redefines (sub)paragraphs to behave more like sections
\ifx\paragraph\undefined\else
\let\oldparagraph\paragraph
\renewcommand{\paragraph}[1]{\oldparagraph{#1}\mbox{}}
\fi
\ifx\subparagraph\undefined\else
\let\oldsubparagraph\subparagraph
\renewcommand{\subparagraph}[1]{\oldsubparagraph{#1}\mbox{}}
\fi

%%% Use protect on footnotes to avoid problems with footnotes in titles
\let\rmarkdownfootnote\footnote%
\def\footnote{\protect\rmarkdownfootnote}

%%% Change title format to be more compact
\usepackage{titling}

% Create subtitle command for use in maketitle
\newcommand{\subtitle}[1]{
  \posttitle{
    \begin{center}\large#1\end{center}
    }
}

\setlength{\droptitle}{-2em}
  \title{Linear Programming}
  \pretitle{\vspace{\droptitle}\centering\huge}
  \posttitle{\par}
  \author{Abhinav Anand}
  \preauthor{\centering\large\emph}
  \postauthor{\par}
  \date{}
  \predate{}\postdate{}

\linespread{1.25}
\usepackage{amsmath}

\begin{document}
\maketitle

\section{Background}\label{background}

Linear programming involves maximizing or minimizing a linear objective
functions subject to linear inequality constraints.

\subsection{Illustration 1:}\label{illustration-1}

We consider a trivial example: suppose the objective function is
\(f(x) = 2x\) and suppose that \(x\) is constrained to be a positive
number not more than 5. More formally,

\[
\text{max } 2x: x\leq 5, x>0
\] While this is clearly an admissible linear program, it's fairly
trivial to solve. Since \(2x\) is linear and monotonic in \(x\), its
solution will occur at the end point of \(x=5\) where it attains its
maximum of 10.

\subsection{Illustration 2:}\label{illustration-2}

For a linear program whose objective function has two variables,
consider a classic portfolio analysis problem: bonds generate 5\%
returns, stocks generate 8\% returns. The total budget is \$1000. How
much of each asset should be bought?

We can translate this problem into a linear programming problem:

\[
\text{max } 0.05b+0.08s: b+s\leq 1000, b \geq 0, s \geq 0
\] where \(b, s\) respectively are the amount (in dollars) invested in
bonds and stocks.

\subsubsection{The Feasible Set}\label{the-feasible-set}

Any combination of bonds and stocks that satisfies the inequalities:
\(b ,s \geq 0\) and \(b+s\leq 1000\) is \emph{feasible}. In the problem
above, the feasible set is the following triangular region:

\begin{Shaded}
\begin{Highlighting}[]
\NormalTok{b <-}\StringTok{ }\DecValTok{0}\OperatorTok{:}\DecValTok{1000}
\NormalTok{s <-}\StringTok{ }\DecValTok{1000} \OperatorTok{-}\StringTok{ }\NormalTok{b}

\KeywordTok{ggplot}\NormalTok{(}\KeywordTok{data.frame}\NormalTok{(}\KeywordTok{cbind}\NormalTok{(b, s)), }\KeywordTok{aes}\NormalTok{(b, s)) }\OperatorTok{+}
\StringTok{  }\KeywordTok{geom_line}\NormalTok{() }\OperatorTok{+}
\StringTok{  }\KeywordTok{geom_vline}\NormalTok{(}\DataTypeTok{xintercept =} \DecValTok{0}\NormalTok{) }\OperatorTok{+}\StringTok{ }\CommentTok{#vertical line}
\StringTok{  }\KeywordTok{geom_hline}\NormalTok{(}\DataTypeTok{yintercept =} \DecValTok{0}\NormalTok{) }\OperatorTok{+}\StringTok{ }\CommentTok{#horizontal line}
\StringTok{  }\KeywordTok{theme_minimal}\NormalTok{()}
\end{Highlighting}
\end{Shaded}

\includegraphics{Linear_Programming_files/figure-latex/plot_feasible-1.pdf}

In the problem above and more generally, in any linear programming
problem, the feasible set is an intersection of \emph{half-spaces}.
Clearly, the more constraints we have, the smaller the feasible set is.
The feasible set in general can be of three varieties:

\begin{enumerate}
\def\labelenumi{\arabic{enumi}.}
\tightlist
\item
  It is empty. In this case there is no solution.
\item
  It is not empty but the objective function is unbounded over it.
  (\(f(x)\in\{\infty,-\infty\}\).)
\item
  It is not empty \emph{and} the objective function is bounded over it.
  (\(f(x)\in (\infty,-\infty)\).)
\end{enumerate}

Only the last case has practical value.

\subsubsection{Finding the Minimum}\label{finding-the-minimum}

In principle, to find the minimum, all we need to do is to evaluate the
objective function at all feasible points; and then see which point
yields the minimum. Clearly, this is not practical since there are a
continuum of points in this case.

The key idea is to consider a sequence of contour lines for the
objective function. Here we consider the family of lines
\(0.05b + 0.08s = \{20,35,50,\hdots\}\) etc. Then we consider which of
these intersect with our feasible set. The maximum of the objective
function will be attained at a \emph{corner point}. (Can you see why?
Hint: See illustration 1.)

\begin{Shaded}
\begin{Highlighting}[]
\NormalTok{y_}\DecValTok{1}\NormalTok{ <-}\StringTok{ }\NormalTok{(}\DecValTok{20}\OperatorTok{-}\FloatTok{0.08}\OperatorTok{*}\NormalTok{s)}\OperatorTok{/}\FloatTok{0.05} \CommentTok{#contour line: .05b+.08s=20}
\NormalTok{y_}\DecValTok{2}\NormalTok{ <-}\StringTok{ }\NormalTok{(}\DecValTok{35}\OperatorTok{-}\FloatTok{0.08}\OperatorTok{*}\NormalTok{s)}\OperatorTok{/}\FloatTok{0.05}
\NormalTok{y_}\DecValTok{3}\NormalTok{ <-}\StringTok{ }\NormalTok{(}\DecValTok{50}\OperatorTok{-}\FloatTok{0.08}\OperatorTok{*}\NormalTok{s)}\OperatorTok{/}\FloatTok{0.05}
\NormalTok{y_}\DecValTok{4}\NormalTok{ <-}\StringTok{ }\NormalTok{(}\DecValTok{65}\OperatorTok{-}\FloatTok{0.08}\OperatorTok{*}\NormalTok{s)}\OperatorTok{/}\FloatTok{0.05}
\NormalTok{y_}\DecValTok{5}\NormalTok{ <-}\StringTok{ }\NormalTok{(}\DecValTok{80}\OperatorTok{-}\FloatTok{0.08}\OperatorTok{*}\NormalTok{s)}\OperatorTok{/}\FloatTok{0.05}

\NormalTok{data_plot_w <-}\StringTok{ }\KeywordTok{cbind}\NormalTok{(b, s, y_}\DecValTok{1}\NormalTok{, }
\NormalTok{                     y_}\DecValTok{2}\NormalTok{, y_}\DecValTok{3}\NormalTok{, }
\NormalTok{                     y_}\DecValTok{4}\NormalTok{, y_}\DecValTok{5}\NormalTok{) }\OperatorTok
\StringTok{  }\NormalTok{dplyr}\OperatorTok{::}\KeywordTok{as_tibble}\NormalTok{() }\CommentTok{#wide format}

\NormalTok{data_plot_l <-}\StringTok{ }\NormalTok{tidyr}\OperatorTok{::}\KeywordTok{gather}\NormalTok{(data_plot_w, }
\NormalTok{                             s}\OperatorTok{:}\NormalTok{y_}\DecValTok{5}\NormalTok{,}
                             \DataTypeTok{key =} \StringTok{"variables"}\NormalTok{,}
                             \DataTypeTok{value =} \StringTok{"obj_fun"}\NormalTok{) }\CommentTok{#long format}

\KeywordTok{ggplot}\NormalTok{(}\DataTypeTok{data =}\NormalTok{ data_plot_l, }\KeywordTok{aes}\NormalTok{(b, obj_fun)) }\OperatorTok{+}
\StringTok{  }\KeywordTok{geom_line}\NormalTok{(}\KeywordTok{aes}\NormalTok{(}\DataTypeTok{linetype =}\NormalTok{ variables)) }\OperatorTok{+}
\StringTok{  }\KeywordTok{geom_vline}\NormalTok{(}\DataTypeTok{xintercept =} \DecValTok{0}\NormalTok{) }\OperatorTok{+}
\StringTok{  }\KeywordTok{geom_hline}\NormalTok{(}\DataTypeTok{yintercept =} \DecValTok{0}\NormalTok{) }\OperatorTok{+}
\StringTok{  }\KeywordTok{theme_minimal}\NormalTok{()}
\end{Highlighting}
\end{Shaded}

\includegraphics{Linear_Programming_files/figure-latex/plot_contour-1.pdf}

This plot suggests that the optimal cannot occur at a strictly interior
point in the feasible set and that it must occur at some corner
point.\footnote{In general the solution could occur along some edge as
  well.} This is simple to see since the contour lines move steadily
upwards until they intersect the feasible set. This yields a tempting
tentative solution which compute the objective function at all (finitely
many) corners and just compares all values to find the optimal. However,
for general problems, there could be several million corner points and
this approach does not scale. Hence we prefer to reach the minimum in a
more systematic way.

\subsection{The Simplex Method}\label{the-simplex-method}

Devised by George Dantzig, the simplex method relies on a simple
insight: look for an optimal solution by starting from a corner and
visiting some accessible corner with lower cost until we reach a corner
for which there is no accessible corner with cost any lower.


\end{document}
